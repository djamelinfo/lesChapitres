\documentclass[10pt,a4paper]{report}
\usepackage[left=25mm, right=25mm, top=20mm, bottom=20mm]{geometry}
\usepackage[utf8]{inputenc}
\usepackage{arabtex}
\usepackage[LFE,LAE]{fontenc}
%\usepackage[arabic]{babel}
\usepackage[farsi,french,arabic]{babel}
\usepackage{pslatex}
\usepackage[math]{kurier}
% \linespread{0.5}

\begin{document}

\renewcommand{\thepage}{}

%\vspace*{3cm}
\hspace*{-5mm}\textbf{\Huge الملخّص}\\\\ 
أصبحت الحوسبة السحابية في السنوات الأخيرة النموذج المهيمن في ساحة تكنولوجيا المعلومات. مبدأها هو تقديم خدمات لامركزية، مما يسمح للعملاء على دفع ما يحتاجون إليه فقط. الطلب المتزايد على هذا النوع من الخدمات يجلب مقدمي خدمة السحابة لزيادة حجم بنيتها التحتية إلى حد أن استهلاك الطاقة والتكاليف المرتبطة بها أصبح أمرا بالغ الأهمية. 
\\
يجب على مزود الخدمة السحابية تلبية مجموعة من مطالب المختلفة. لهذا السبب نحن مهتمون في هذه الوثيقة على إدارة استهلاك الطاقة في مجال الحوسبة السحابية. وقمنا بتقديم نهجنا الذي يقلل من استهلاك الطاقة وعدد انتهاكات اتفاقية مستوى الخدمة. ونتيجة لذلك تقليل من انبعاث الحراري.
\\\\
{الكلمات المفاتيح:   الحوسبة السحابية، مركز البيانات، الاستثمار، آلة افتراضية، هجرة، الطاقة}.
%
\\

\vspace*{3cm}
\textLR{\Huge \textbf{Résumé}\\\\
\normalsize
L’informatique dans les nuages (Cloud Computing) est devenu durant les dernières années un paradigme dominant dans le paysage informatique. Son principe est de fournir des services décentralisés à la demande et de permettre aux clients de payer uniquement les services dont ils auront besoin. La demande croissante pour ce type de service amène les fournisseurs de service Clouds à augmenter la taille de leurs infrastructures à tel point que les consommations d’énergie ainsi que les coûts associés deviennent très importants.\\
Chaque fournisseur de service cloud doit répondre à des demandes différentes. Les gestionnaires d’infrastructure doivent héberger un ensemble de ces services. C’est pourquoi au cours de ce travail, nous nous sommes intéressés à la gestion de la consommation d’énergie dans le Cloud Computing, où nous avons présenté nos approches qui permettent de réduire la consommation d’énergie et le nombre de violations de SLA et par conséquent minimiser le dégagement de chaleur.  \\
}

\textLR{\textbf{Mots clés}: Cloud computing, Data center, Placement, Virtual machine, Migration, énergie.
\\
}
\\

\vspace*{3cm}
\textLR{ {\centering \Huge \textbf{Abstract}}\\\\
\normalsize
Cloud computing has become over the last years an important paradigm in the computing landscape. Its principle is to provide decentralized services and allows client to consume resources on a pay-as-you-go model. The growing demand for this type of service brings providers service clouds to increase the size of their infrastructure to the point that the energy consumption and associated costs become very important\\
Each cloud service provider has to provide different types of requests. Infrastructure managers must accommodate a range of these services. That is why in this memory we are interested in the management of energy consumption in cloud computing, where we presented our approaches which allow to reduce energy consumption and the number of SLA violations and thus minimize the exotherm.\\
}

\textLR{\textbf{Keywords}: Cloud computing, Data center, Placing, Virtual machine, migration, Energy.\\
}




\end{document}
