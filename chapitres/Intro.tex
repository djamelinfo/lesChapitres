\thispagestyle{empty}
\begin{center}
\resizebox{!}{8mm}{\bf \scshape \selectfont Introduction générale}\\
\end{center}
%\lhead{\normalsize \bf \selectfont Introduction générale}
\vspace{20mm}

%\lipsum[1-3]
\begin{onehalfspace}
\lettrine[nindent=1em,lines=3]{L}'informatique dans les nuages (cloud computing en anglais) s’est imposée ces dernières années comme un paradigme majeur d’utilisation des infrastructures informatiques. Celui-ci répond à des besoins et demandes croissantes en terme de disponibilité et flexibilité. 
Le développement remarquable du Cloud Computing, ces dernières années, suscite de plus en plus l’intérêt des différents utilisateurs de l’Internet et de l’informatique qui cherchent à profiter au mieux des services et des applications disponibles en ligne à travers le Web en mode services à la demande et facturation à l’usage.\\

L’approche du Cloud Computing s’appuie principalement sur le concept de virtualisation. Ce concept est un ensemble de techniques permettant de faire fonctionner sur une seule machine plusieurs systèmes d’exploitation et/ou plusieurs applications, isolés les uns des autres. Un Cloud est constitué d’un ensemble de machines virtuelles qui utilisent la même infrastructure physique.\\

La croissance des deux côtés de l’offre et de la demande rend le problème de la consommation d’énergie plus complexe, sophistiqué dans un environnement Cloud. Les demandes cumulées sur une seule machine virtuelle ou en général sur un seul Data Center mènent à la saturation. Ces derniers ne pourront plus satisfaire les demandes des utilisateurs. D’autre part les besoins en ressources des applications peuvent aller au-delà de ce qui est disponible dans un Cloud.\\

La consommation électrique des centres de données (data centers en anglais) dans le monde a été estimée pour 2010 entre 1.1\% et 1.5\% de la consommation électrique globale. Même si cela parait peu en proportion, cette consommation d’énergie représente plus de 200 milliards de kilowatt-heure \cite{ref48}. Cette consommation d’énergie est en croissance permanente au fil des années. Elle était estimée à 12\% par an en 2007 par l’U.S Environmental Protection Agency pour atteindre un coût de l’électricité utilisée par les serveurs des data centers estimé de 7.4 milliards de dollars \cite{EPA_2007}.\\

Dans ce travail, l’objectif visé est de proposer une stratégie basée sur la migration des machines virtuelles, en appliquant des méthodes et des algorithmes. Notre politique de migration des machines virtuelles vise à optimiser la consommation d’énergie.\\

{\scshape\bfseries Organisation du mémoire}\\

Le présent mémoire est structuré autour de quatre principaux chapitres qui se résument comme suit :
\begin{description}
\item[Chapitre 1 :] Dans le premier chapitre, nous présenterons le problème de la consommation d’énergie des data center et son impact énergétique.
\item[Chapitre 2 :] Le second chapitre présentera quelques différentes techniques d’optimisations d’énergie et les principaux travaux de recherches qui ont été proposés dans la littérature.
\item[Chapitre 3 :] Le troisième chapitre sera réservé à la description détaillée de la conception de la stratégie utilisée ainsi que des services que nous avons proposés. Cette conception se fera à l’aide de formules, d’algorithmes et de diagrammes UML.
\item[Chapitre 4 :] Ce dernier chapitre présentera les étapes de l’implémentation de l’approche proposée. Nous y détaillerons la réalisation de certaines fonctionnalités ainsi que l’étude d’évaluation de cette stratégie. Les résultats d’expérimentation seront interprétés.
\end{description}

Enfin, un ensemble de perspectives viendra clore notre travail.

\end{onehalfspace}
