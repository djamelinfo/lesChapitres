\thispagestyle{empty}
\begin{center}
\resizebox{!}{8mm}{\bf \scshape \selectfont Conclusion générale}\\
\end{center}
%\lhead{\normalsize \bf \selectfont Introduction générale}
\vspace{20mm}

%\lipsum[1-3]
\begin{onehalfspace}
\lettrine[nindent=1em,lines=3]{L}e cloud computing est en pleine expansion et tend à s'imposer comme un des paradigmes dominants dans l'univers informatique. Les infrastructures proposant des services de cloud computing deviennent donc de plus en plus nombreuses, et de plus en plus complexes pour répondre à cette demande croissante de services décentralisés. Cette augmentation amène bien évidemment divers problèmes, dont celui de la consommation d'énergie. Il faut donc concevoir des techniques et outils afin de répondre à ces nouveaux besoins de gestion.\\

La migration des machines virtuelles est une technique qui permet de régler en général certains problèmes dans le Cloud tels que les problèmes de la consommation d'énergie.\\

Cette technique consiste à déplacer une machine virtuelle d'un nœud à un autre dont le but d'augmenter considérablement le taux d'utilisation et réduire la consommation d'énergie selon certains critères.\\

Au cours de ce projet, nous avons développé une stratégie pour bien améliorer l'efficacité énergétique dans le Cloud Computing. Nous avons proposé une approche qui se base sur le mécanisme de la migration des machines virtuelles, tout en appliquant certaines méthodes. Et afin de mettre en évidence l'approche proposée, nous avons réalisé plusieurs séries d'expérimentations en faisant varier plusieurs paramètres. Nous avons utilisé, également, les trois principales métriques qui sont l'énergie consommée, nombre de migrations et violations de SLA Sur le plan d'expérimentation, nous avons positionné et comparé nos propositions par rapport à l'approche sans migration  et à l'approche \textit{Single Threshold}.
\clearpage
\textbf{\scshape Perspectives}\\\\
Pour une continuation de notre travail, plusieurs perspectives peuvent être envisagées :
\begin{enumerate}
\item Ajouter un seuil de température afin de limiter  la chaleur dégagée des machines physiques et de minimiser l'énergie consommée des serveurs et des systèmes de refroidissement.
\item Étudier l'influence des systèmes de climatisation des data center sur leur efficacités et leur performances.
\item Implémenter les deux approches proposées dans un environnement de Cloud réel.
\item En plus de l'utilisation du processeur,  prendre en considération   plusieurs ressources au niveau de la phase de migration tels que: la capacité de la RAM, la capacité de stockage et la bande passante. 
\end{enumerate}

\end{onehalfspace}
